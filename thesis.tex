\documentclass[12pt]{article}
\usepackage{amsmath}
\usepackage{amssymb}
\usepackage{graphicx}
\usepackage[margin=1in]{geometry}
\newcommand{\del}{\nabla}
\begin{document}

\title{Nonlinear Analysis of VR Synchronization}
\author{Howon Lee}
\maketitle

\section{Introduction}
Theory of synchronization in VR-land (communications land), theory of synchronization in physics-land, aka nonlinear science land. Why do both exist and have not met each other? Have there been any other contacts?

Talk about the physiology of heart rate variability, how it was inspirational.

Time series analysis and dynamical systems analysis exist. Here, we will do both. Why?

\section{Time Domain Analysis}
Correlation and the cornettos go here
\section{Frequency Domain Analysis}
Hilbert space and fourier transform methods go here

\section{Phase Transition to Synchronization}
Mutual information analysis goes here. So does a KS entropy analysis

Create the state space embedding, talk about the method of the creation of state space embedding

There might be a phase transition to synchronization: arnold maps go here

\section{Deterministic dual between discrete time series and ordered multigraph}

There's a fairly big previous literature on this. Visibility graph, dual graph, spectral methods, recurrence graph methods, steal forward citations for this.

What's the ordered multigraph that I'm talking about? Mathematical definition. Talk about the construction of the multigraph from the time series.

Why do you need a discrete time series? How are we discretizing our time series now? Why is it a deterministic dual, what is the algorithm and process by which we create this deterministic dual: that is, how to get from multigraph to time series?

Recreate the time series of logistic map here.

Note tha time series resampling is now possible deterministically. Do it with the logistic map here.

Do the time series resampling with the damned other things.

Now, do complex network analysis things with the VR data and with logistic map.

\end{document}
