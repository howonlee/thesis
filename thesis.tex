\documentclass[12pt]{article}
\usepackage{amsmath}
\usepackage{amssymb}
\usepackage{graphicx}
\usepackage[margin=1in]{geometry}
\newcommand{\del}{\nabla}
\begin{document}

\title{Nonlinear Analysis of VR Synchronization}
\author{Howon Lee}
\maketitle

\section{Abstract}
There exists a literature dealing with time series analysis in the nonlinear sciences which in the case of synchronization. We attempt many of these methods in analyzing an instance of the phenomenon of interpersonal synchrony in a virtual environment.%%% We also propose a mapping from time series to networks with a deterministic inverse.

\section{Introduction}

%%% probably become more familiar with the social sciency things here

%%% what disciplines exist that deal with this? there are a lot.

%%% talk about the non-automatic tools that deal with this.

%%% why automatic tools?

%%% what automatic tools exist?

%%% what's the scopy of this project?

%%% definitions?

%%% what are the functions of synchrony?
Interpersonal synchrony is defined in the social psychology literature as individuals' termporal coordination during social interaction. In the physical sciences, synchronization is defined as an adjustment of rhythms of oscillating objects due to weak interaction, with generalizations possible for chaotic systems, which depend upon a phase of the chaotic system existing. %%% cite social psych, physics definition of synchronization

Given that the signals created by an individual during social interaction have a phase, and social interaction can be construed as a weak interaction between the two individual systems, it should be clear that the definition of interpersonal synchrony given in the social psychological literature is a subset of the definition given in the physics literature. This has often been noted, and has therefore spawned a cross-disciplinary field wherein one uses signals processing techniques to measure interpersonal synchrony. %%% cite the reviews of this thing

%%% correlation (time-lagged cross-correlation)
%%% recurrence analysis
%%% spectral methods
%%% review MI method in physics literature.

\section{Time Domain Analysis}
Correlation and the cornettos go here. Talk about benefits and detriments of the cornetto stuff
Mutual information analysis goes here. So does a KS entropy analysis

\section{Frequency Domain Analysis}
Hilbert space and fourier transform methods go here. The \emph{really} exciting thing about these analyses are that they allow the study of the phase of the thing

\section{Phase Transition to Synchronization}

Create the state space embedding, talk about the method of the creation of state space embedding

There might be a phase transition to synchronization: arnold maps go here

%\section{Deterministic dual between discrete time series and ordered multigraph}
%
%There's a fairly big previous literature on this. Visibility graph, dual graph, spectral methods, recurrence graph methods, steal forward citations for this.
%
%What's the ordered multigraph that I'm talking about? Mathematical definition. Talk about the construction of the multigraph from the time series.
%
%Why do you need a discrete time series? How are we discretizing our time series now? Why is it a deterministic dual, what is the algorithm and process by which we create this deterministic dual: that is, how to get from multigraph to time series?
%
%Recreate the time series of logistic map here.
%
%Note tha time series resampling is now possible deterministically. Do it with the logistic map here.
%
%Do the time series resampling with the damned other things.
%
%Now, do complex network analysis things with the VR data and with logistic map.

\end{document}
